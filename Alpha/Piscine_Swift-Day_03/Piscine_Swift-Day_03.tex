%%%%%%%%%%%%  Generated using docx2latex.com  %%%%%%%%%%%%%%

%%%%%%%%%%%%  v2.0.0-beta  %%%%%%%%%%%%%%

\documentclass[12pt]{report}
\usepackage{amsmath}
\usepackage{latexsym}
\usepackage{amsfonts}
\usepackage[normalem]{ulem}
\usepackage{array}
\usepackage{amssymb}
\usepackage{graphicx}
\usepackage[backend=biber,
style=numeric,
sorting=none,
isbn=false,
doi=false,
url=false,
]{biblatex}\addbibresource{bibliography.bib}

\usepackage{subfig}
\usepackage{wrapfig}
\usepackage{wasysym}
\usepackage{enumitem}
\usepackage{adjustbox}
\usepackage{ragged2e}
\usepackage[svgnames,table]{xcolor}
\usepackage{tikz}
\usepackage{longtable}
\usepackage{changepage}
\usepackage{setspace}
\usepackage{hhline}
\usepackage{multicol}
\usepackage{tabto}
\usepackage{float}
\usepackage{multirow}
\usepackage{makecell}
\usepackage{fancyhdr}
\usepackage[toc,page]{appendix}
\usepackage[hidelinks]{hyperref}
\usetikzlibrary{shapes.symbols,shapes.geometric,shadows,arrows.meta}
\tikzset{>={Latex[width=1.5mm,length=2mm]}}
\usepackage{flowchart}\usepackage[paperheight=11.0in,paperwidth=8.5in,left=0.5in,right=0.5in,top=0.5in,bottom=0.5in,headheight=1in]{geometry}
\usepackage[utf8]{inputenc}
\usepackage[T1]{fontenc}
\TabPositions{0.5in,1.0in,1.5in,2.0in,2.5in,3.0in,3.5in,4.0in,4.5in,5.0in,5.5in,6.0in,6.5in,7.0in,}

\urlstyle{same}


 %%%%%%%%%%%%  Set Depths for Sections  %%%%%%%%%%%%%%

% 1) Section
% 1.1) SubSection
% 1.1.1) SubSubSection
% 1.1.1.1) Paragraph
% 1.1.1.1.1) Subparagraph


\setcounter{tocdepth}{5}
\setcounter{secnumdepth}{5}


 %%%%%%%%%%%%  Set Depths for Nested Lists created by \begin{enumerate}  %%%%%%%%%%%%%%


\setlistdepth{9}
\renewlist{enumerate}{enumerate}{9}
		\setlist[enumerate,1]{label=\arabic*)}
		\setlist[enumerate,2]{label=\alph*)}
		\setlist[enumerate,3]{label=(\roman*)}
		\setlist[enumerate,4]{label=(\arabic*)}
		\setlist[enumerate,5]{label=(\Alph*)}
		\setlist[enumerate,6]{label=(\Roman*)}
		\setlist[enumerate,7]{label=\arabic*}
		\setlist[enumerate,8]{label=\alph*}
		\setlist[enumerate,9]{label=\roman*}

\renewlist{itemize}{itemize}{9}
		\setlist[itemize]{label=$\cdot$}
		\setlist[itemize,1]{label=\textbullet}
		\setlist[itemize,2]{label=$\circ$}
		\setlist[itemize,3]{label=$\ast$}
		\setlist[itemize,4]{label=$\dagger$}
		\setlist[itemize,5]{label=$\triangleright$}
		\setlist[itemize,6]{label=$\bigstar$}
		\setlist[itemize,7]{label=$\blacklozenge$}
		\setlist[itemize,8]{label=$\prime$}

\setlength{\topsep}{0pt}\setlength{\parindent}{0pt}

 %%%%%%%%%%%%  This sets linespacing (verticle gap between Lines) Default=1 %%%%%%%%%%%%%%


\renewcommand{\arraystretch}{1.3}


%%%%%%%%%%%%%%%%%%%% Document code starts here %%%%%%%%%%%%%%%%%%%%



\begin{document}

\vspace{\baselineskip}

\vspace{\baselineskip}

\vspace{\baselineskip}

\vspace{\baselineskip}

\vspace{\baselineskip}
\par

\section*{InstaFilter}
\addcontentsline{toc}{section}{InstaFilter}

\vspace{\baselineskip}

\vspace{\baselineskip}
\paragraph*{Michael BRAVE }
\addcontentsline{toc}{paragraph}{Michael BRAVE }
\paragraph*{42 Staff }
\addcontentsline{toc}{paragraph}{42 Staff }

\vspace{\baselineskip}

\vspace{\baselineskip}

\vspace{\baselineskip}

\vspace{\baselineskip}
\begin{Center}
\textit{Summary: This document contains the subject for Day 03 for the $``$Piscine Swift$"$  from 42}
\end{Center}\par


\vspace{\baselineskip}

\vspace{\baselineskip}

\vspace{\baselineskip}

\vspace{\baselineskip}

\vspace{\baselineskip}

\vspace{\baselineskip}

\vspace{\baselineskip}

\vspace{\baselineskip}


 %%%%%%%%%%%%  Starting New Page here %%%%%%%%%%%%%%

\newpage

\vspace{\baselineskip}
\vspace{\baselineskip}
\section*{Contents}
\addcontentsline{toc}{section}{Contents}

\vspace{\baselineskip}
\subsubsection*{I\hspace*{10pt}\hspace*{10pt}Foreword}
\addcontentsline{toc}{subsubsection}{I\hspace*{10pt}\hspace*{10pt}Foreword}
\subsubsection*{II\hspace*{10pt}\hspace*{10pt}General Instructions}
\addcontentsline{toc}{subsubsection}{II\hspace*{10pt}\hspace*{10pt}General Instructions}
\subsubsection*{III\hspace*{10pt}\hspace*{10pt}Introduction}
\addcontentsline{toc}{subsubsection}{III\hspace*{10pt}\hspace*{10pt}Introduction}
\subsubsection*{IV\hspace*{10pt}\hspace*{10pt}Exercise 00: Pictures}
\addcontentsline{toc}{subsubsection}{IV\hspace*{10pt}\hspace*{10pt}Exercise 00: Pictures}
\subsubsection*{V\hspace*{10pt}\hspace*{10pt}Exercise 01: Multi Touch}
\addcontentsline{toc}{subsubsection}{V\hspace*{10pt}\hspace*{10pt}Exercise 01: Multi Touch}
\subsubsection*{VI\hspace*{10pt}\hspace*{10pt}Exercise 02: Stretch $\&$  Zoom}
\addcontentsline{toc}{subsubsection}{VI\hspace*{10pt}\hspace*{10pt}Exercise 02: Stretch $\&$  Zoom}
\subsubsection*{VII\hspace*{10pt}\hspace*{10pt}Exercise 03: Filters}
\addcontentsline{toc}{subsubsection}{VII\hspace*{10pt}\hspace*{10pt}Exercise 03: Filters}
\subsubsection*{VIII\hspace*{10pt}\hspace*{10pt}Exercise 04: Carousel}
\addcontentsline{toc}{subsubsection}{VIII\hspace*{10pt}\hspace*{10pt}Exercise 04: Carousel}
\subsubsection*{XI\hspace*{10pt}\hspace*{10pt}Bonus: Save Changes To Photos}
\addcontentsline{toc}{subsubsection}{XI\hspace*{10pt}\hspace*{10pt}Bonus: Save Changes To Photos}

\vspace{\baselineskip}

\vspace{\baselineskip}

\vspace{\baselineskip}

\vspace{\baselineskip}

\vspace{\baselineskip}

\vspace{\baselineskip}


 %%%%%%%%%%%%  Starting New Page here %%%%%%%%%%%%%%

\newpage

\vspace{\baselineskip}
\vspace{\baselineskip}
\section*{Chapter I}
\addcontentsline{toc}{section}{Chapter I}
\section*{Foreword}
\addcontentsline{toc}{section}{Foreword}
Some quotes about photos and photography\par


\vspace{\baselineskip}
$``$Photography is about finding out what can happen in the frame. When you put four edges around some facts, you change those facts.$"$ \par

– Garry Winogrand\par


\vspace{\baselineskip}
$``$I began to realize that the camera sees the world differently than the human eye and that sometimes those differences can make a photograph more powerful than what you actually observed.$"$ \par

– Galen Rowell\par


\vspace{\baselineskip}
$``$It’s a successful photo when I would rather look at it than the subject matter$"$ \par

-Anonymous\par



 %%%%%%%%%%%%  Starting New Page here %%%%%%%%%%%%%%

\newpage

\vspace{\baselineskip}
\vspace{\baselineskip}
\section*{Chapter II}
\addcontentsline{toc}{section}{Chapter II}
\section*{General Instructions}
\addcontentsline{toc}{section}{General Instructions}
\begin{itemize}
	\item Only this document will serve as reference. Do not trust rumors.\par

	\item Read carefully the whole subject before beginning.\par

	\item Watch out! This document could potentially change up to an hour before submission.\par

	\item This project will be corrected by humans only.\par

	\item This course is designed to build on previous days’ concepts, try your hardest to finish everyday.\par

	\item Each day culminates in a portfolio piece, if you finish the day this is something you can use to get hired.\par

	\item When submitting, submit the folder of the Xcode project.\par

	\item Only the work submitted on the repository will be accounted for during peer-2-peer correction.\par

	\item Here it is the \href{https://docs.swift.org/swift-book/}{\textcolor[HTML]{1155CC}{\uline{official manual of Swift}}} and the \href{https://developer.apple.com/documentation/swift/swift_standard_library}{\textcolor[HTML]{1155CC}{\uline{Swift Standard Library}}}\par

	\item It is forbidden to use other libraries, packages, pods, etc. Unless otherwise stated in the project.\par

	\item Got a question? Ask your peer on the right. Otherwise, try your peer on the left.\par

	\item You can discuss on the Piscine forum of your Intra!\par

	\item By Odin, by Thor! Use your brain!!!
\end{itemize}\par


\vspace{\baselineskip}

\vspace{\baselineskip}


 %%%%%%%%%%%%  Starting New Page here %%%%%%%%%%%%%%

\newpage

\vspace{\baselineskip}
\vspace{\baselineskip}
\section*{Chapter III}
\addcontentsline{toc}{section}{Chapter III}
\section*{Introduction}
\addcontentsline{toc}{section}{Introduction}
Today we are learning how to use and manipulate photos. For this we will also be learning about multi touch gestures, and Scene transition animations. \par


\vspace{\baselineskip}


 %%%%%%%%%%%%  Starting New Page here %%%%%%%%%%%%%%

\newpage

\vspace{\baselineskip}
\vspace{\baselineskip}
\section*{Chapter IV}
\addcontentsline{toc}{section}{Chapter IV}
\section*{Exercise 00 : Pictures}
\addcontentsline{toc}{section}{Exercise 00 : Pictures}

\vspace{\baselineskip}

\vspace{\baselineskip}

\vspace{\baselineskip}


%%%%%%%%%%%%%%%%%%%% Table No: 1 starts here %%%%%%%%%%%%%%%%%%%%


\begin{table}[H]
 			\centering
\begin{tabular}{p{7.3in}}
\hline
%row no:1
\multicolumn{1}{|p{7.3in}|}{\Centering Exercise : 00} \\
\hhline{-}
%row no:2
\multicolumn{1}{|p{7.3in}|}{\Centering Pictures} \\
\hhline{-}
%row no:3
\multicolumn{1}{|p{7.3in}|}{Files to turn in: .xcodeproj and all necessary files} \\
\hhline{-}
%row no:4
\multicolumn{1}{|p{7.3in}|}{Allowed functions : Swift Standard Library, UIKit, UIImagePickerController} \\
\hhline{-}
%row no:5
\multicolumn{1}{|p{7.3in}|}{Notes : n/a} \\
\hhline{-}

\end{tabular}
 \end{table}


%%%%%%%%%%%%%%%%%%%% Table No: 1 ends here %%%%%%%%%%%%%%%%%%%%


\vspace{\baselineskip}

\vspace{\baselineskip}
We will create an app that will access and load multiple images stored in device memory. List photos by name, and on open display image full screen. \par


\vspace{\baselineskip}


 %%%%%%%%%%%%  Starting New Page here %%%%%%%%%%%%%%

\newpage

\vspace{\baselineskip}
\vspace{\baselineskip}
\section*{Chapter V}
\addcontentsline{toc}{section}{Chapter V}
\section*{Exercise 01 : Multi Touch}
\addcontentsline{toc}{section}{Exercise 01 : Multi Touch}

\vspace{\baselineskip}

\vspace{\baselineskip}

\vspace{\baselineskip}


%%%%%%%%%%%%%%%%%%%% Table No: 2 starts here %%%%%%%%%%%%%%%%%%%%


\begin{table}[H]
 			\centering
\begin{tabular}{p{7.3in}}
\hline
%row no:1
\multicolumn{1}{|p{7.3in}|}{\Centering Exercise : 01} \\
\hhline{-}
%row no:2
\multicolumn{1}{|p{7.3in}|}{\Centering Multi Touch} \\
\hhline{-}
%row no:3
\multicolumn{1}{|p{7.3in}|}{Files to turn in: .xcodeproj and all necessary files} \\
\hhline{-}
%row no:4
\multicolumn{1}{|p{7.3in}|}{Allowed functions : Swift Standard Library, UIKit} \\
\hhline{-}
%row no:5
\multicolumn{1}{|p{7.3in}|}{Notes : n/a} \\
\hhline{-}

\end{tabular}
 \end{table}


%%%%%%%%%%%%%%%%%%%% Table No: 2 ends here %%%%%%%%%%%%%%%%%%%%


\vspace{\baselineskip}

\vspace{\baselineskip}
Now we want to use the swiping functionality to transition from one image to the next. Left swipe to descend(last) on the list, right swipe to ascend(next). \par



 %%%%%%%%%%%%  Starting New Page here %%%%%%%%%%%%%%

\newpage

\vspace{\baselineskip}
\vspace{\baselineskip}
\section*{Chapter VI}
\addcontentsline{toc}{section}{Chapter VI}
\section*{Exercise 02: Stretch $\&$  Zoom}
\addcontentsline{toc}{section}{Exercise 02: Stretch $\&$  Zoom}

\vspace{\baselineskip}

\vspace{\baselineskip}

\vspace{\baselineskip}


%%%%%%%%%%%%%%%%%%%% Table No: 3 starts here %%%%%%%%%%%%%%%%%%%%


\begin{table}[H]
 			\centering
\begin{tabular}{p{7.3in}}
\hline
%row no:1
\multicolumn{1}{|p{7.3in}|}{\Centering Exercise : 02} \\
\hhline{-}
%row no:2
\multicolumn{1}{|p{7.3in}|}{\Centering Stretch $\&$  Zoom} \\
\hhline{-}
%row no:3
\multicolumn{1}{|p{7.3in}|}{Files to turn in: .xcodeproj and all necessary files} \\
\hhline{-}
%row no:4
\multicolumn{1}{|p{7.3in}|}{Allowed functions : Swift Standard Library, UIKit} \\
\hhline{-}
%row no:5
\multicolumn{1}{|p{7.3in}|}{Notes : n/a} \\
\hhline{-}

\end{tabular}
 \end{table}


%%%%%%%%%%%%%%%%%%%% Table No: 3 ends here %%%%%%%%%%%%%%%%%%%%


\vspace{\baselineskip}
Now we are using other multi touch functions, pinch/double tap to zoom in, spread/double tap to zoom out. We should also be able to navigate while zoomed in (I.e. dragging left will let us see the left side of the document).\par



 %%%%%%%%%%%%  Starting New Page here %%%%%%%%%%%%%%

\newpage

\vspace{\baselineskip}
\vspace{\baselineskip}
\section*{Chapter VII}
\addcontentsline{toc}{section}{Chapter VII}
\section*{Exercise 03: Filters}
\addcontentsline{toc}{section}{Exercise 03: Filters}

\vspace{\baselineskip}

\vspace{\baselineskip}

\vspace{\baselineskip}


%%%%%%%%%%%%%%%%%%%% Table No: 4 starts here %%%%%%%%%%%%%%%%%%%%


\begin{table}[H]
 			\centering
\begin{tabular}{p{7.3in}}
\hline
%row no:1
\multicolumn{1}{|p{7.3in}|}{\Centering Exercise : 03} \\
\hhline{-}
%row no:2
\multicolumn{1}{|p{7.3in}|}{\Centering Filters} \\
\hhline{-}
%row no:3
\multicolumn{1}{|p{7.3in}|}{Files to turn in: .xcodeproj and all necessary files} \\
\hhline{-}
%row no:4
\multicolumn{1}{|p{7.3in}|}{Allowed functions : Swift Standard Library, UIKit, CoreImage} \\
\hhline{-}
%row no:5
\multicolumn{1}{|p{7.3in}|}{Notes : n/a} \\
\hhline{-}

\end{tabular}
 \end{table}


%%%%%%%%%%%%%%%%%%%% Table No: 4 ends here %%%%%%%%%%%%%%%%%%%%


\vspace{\baselineskip}
Now we are applying filters to the images, there should be at least 5 filters to choose from that can be selected with a button each. Then we should display what the photo looks like with the filter applied. \par


\vspace{\baselineskip}
Hint: CIFilter(s)\par


\vspace{\baselineskip}

\vspace{\baselineskip}

\vspace{\baselineskip}


 %%%%%%%%%%%%  Starting New Page here %%%%%%%%%%%%%%

\newpage

\vspace{\baselineskip}
\vspace{\baselineskip}

\vspace{\baselineskip}
\section*{Chapter VIII}
\addcontentsline{toc}{section}{Chapter VIII}
\section*{Exercise 04: Carousel}
\addcontentsline{toc}{section}{Exercise 04: Carousel}

\vspace{\baselineskip}

\vspace{\baselineskip}

\vspace{\baselineskip}


%%%%%%%%%%%%%%%%%%%% Table No: 5 starts here %%%%%%%%%%%%%%%%%%%%


\begin{table}[H]
 			\centering
\begin{tabular}{p{7.3in}}
\hline
%row no:1
\multicolumn{1}{|p{7.3in}|}{\Centering Exercise : 04} \\
\hhline{-}
%row no:2
\multicolumn{1}{|p{7.3in}|}{\Centering Carousel} \\
\hhline{-}
%row no:3
\multicolumn{1}{|p{7.3in}|}{Files to turn in: .xcodeproj and all necessary files} \\
\hhline{-}
%row no:4
\multicolumn{1}{|p{7.3in}|}{Allowed functions : Swift Standard Library, UIKit, UIScrollView, PageControl, ICaro} \\
\hhline{-}
%row no:5
\multicolumn{1}{|p{7.3in}|}{Notes : n/a} \\
\hhline{-}

\end{tabular}
 \end{table}


%%%%%%%%%%%%%%%%%%%% Table No: 5 ends here %%%%%%%%%%%%%%%%%%%%


\vspace{\baselineskip}
We want to be able to transition with a smooth animation from photo to photo. We will do this by creating a carousel style animation to transition from photo to photo. \par


\vspace{\baselineskip}


 %%%%%%%%%%%%  Starting New Page here %%%%%%%%%%%%%%

\newpage

\vspace{\baselineskip}
\vspace{\baselineskip}

\vspace{\baselineskip}

\vspace{\baselineskip}

\vspace{\baselineskip}
\section*{Chapter XI}
\addcontentsline{toc}{section}{Chapter XI}
\section*{Bonus : Save Changes To Photos}
\addcontentsline{toc}{section}{Bonus : Save Changes To Photos}

\vspace{\baselineskip}

\vspace{\baselineskip}

\vspace{\baselineskip}


%%%%%%%%%%%%%%%%%%%% Table No: 6 starts here %%%%%%%%%%%%%%%%%%%%


\begin{table}[H]
 			\centering
\begin{tabular}{p{7.3in}}
\hline
%row no:1
\multicolumn{1}{|p{7.3in}|}{\Centering Bonus} \\
\hhline{-}
%row no:2
\multicolumn{1}{|p{7.3in}|}{\Centering Save Changes To Photos} \\
\hhline{-}
%row no:3
\multicolumn{1}{|p{7.3in}|}{Files to turn in: .xcodeproj and all necessary files} \\
\hhline{-}
%row no:4
\multicolumn{1}{|p{7.3in}|}{Allowed functions : Swift Standard Library, UIKit} \\
\hhline{-}
%row no:5
\multicolumn{1}{|p{7.3in}|}{Notes : n/a} \\
\hhline{-}

\end{tabular}
 \end{table}


%%%%%%%%%%%%%%%%%%%% Table No: 6 ends here %%%%%%%%%%%%%%%%%%%%


\vspace{\baselineskip}
Now we need to be able to take the changes we have made to a photo and to be able to save it to disk. \par


\vspace{\baselineskip}

\printbibliography
\end{document}