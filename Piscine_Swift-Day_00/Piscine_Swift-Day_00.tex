%%%%%%%%%%%%  Generated using docx2latex.com  %%%%%%%%%%%%%%

%%%%%%%%%%%%  v2.0.0-beta  %%%%%%%%%%%%%%

\documentclass[12pt]{report}
\usepackage{amsmath}
\usepackage{latexsym}
\usepackage{amsfonts}
\usepackage[normalem]{ulem}
\usepackage{array}
\usepackage{amssymb}
\usepackage{graphicx}
\usepackage[backend=biber,
style=numeric,
sorting=none,
isbn=false,
doi=false,
url=false,
]{biblatex}\addbibresource{bibliography.bib}

\usepackage{subfig}
\usepackage{wrapfig}
\usepackage{wasysym}
\usepackage{enumitem}
\usepackage{adjustbox}
\usepackage{ragged2e}
\usepackage[svgnames,table]{xcolor}
\usepackage{tikz}
\usepackage{longtable}
\usepackage{changepage}
\usepackage{setspace}
\usepackage{hhline}
\usepackage{multicol}
\usepackage{tabto}
\usepackage{float}
\usepackage{multirow}
\usepackage{makecell}
\usepackage{fancyhdr}
\usepackage[toc,page]{appendix}
\usepackage[hidelinks]{hyperref}
\usetikzlibrary{shapes.symbols,shapes.geometric,shadows,arrows.meta}
\tikzset{>={Latex[width=1.5mm,length=2mm]}}
\usepackage{flowchart}\usepackage[paperheight=11.0in,paperwidth=8.5in,left=0.5in,right=0.5in,top=0.5in,bottom=0.5in,headheight=1in]{geometry}
\usepackage[utf8]{inputenc}
\usepackage[T1]{fontenc}
\TabPositions{0.5in,1.0in,1.5in,2.0in,2.5in,3.0in,3.5in,4.0in,4.5in,5.0in,5.5in,6.0in,6.5in,7.0in,}

\urlstyle{same}


 %%%%%%%%%%%%  Set Depths for Sections  %%%%%%%%%%%%%%

% 1) Section
% 1.1) SubSection
% 1.1.1) SubSubSection
% 1.1.1.1) Paragraph
% 1.1.1.1.1) Subparagraph


\setcounter{tocdepth}{5}
\setcounter{secnumdepth}{5}


 %%%%%%%%%%%%  Set Depths for Nested Lists created by \begin{enumerate}  %%%%%%%%%%%%%%


\setlistdepth{9}
\renewlist{enumerate}{enumerate}{9}
		\setlist[enumerate,1]{label=\arabic*)}
		\setlist[enumerate,2]{label=\alph*)}
		\setlist[enumerate,3]{label=(\roman*)}
		\setlist[enumerate,4]{label=(\arabic*)}
		\setlist[enumerate,5]{label=(\Alph*)}
		\setlist[enumerate,6]{label=(\Roman*)}
		\setlist[enumerate,7]{label=\arabic*}
		\setlist[enumerate,8]{label=\alph*}
		\setlist[enumerate,9]{label=\roman*}

\renewlist{itemize}{itemize}{9}
		\setlist[itemize]{label=$\cdot$}
		\setlist[itemize,1]{label=\textbullet}
		\setlist[itemize,2]{label=$\circ$}
		\setlist[itemize,3]{label=$\ast$}
		\setlist[itemize,4]{label=$\dagger$}
		\setlist[itemize,5]{label=$\triangleright$}
		\setlist[itemize,6]{label=$\bigstar$}
		\setlist[itemize,7]{label=$\blacklozenge$}
		\setlist[itemize,8]{label=$\prime$}

\setlength{\topsep}{0pt}\setlength{\parindent}{0pt}

 %%%%%%%%%%%%  This sets linespacing (verticle gap between Lines) Default=1 %%%%%%%%%%%%%%


\renewcommand{\arraystretch}{1.3}


%%%%%%%%%%%%%%%%%%%% Document code starts here %%%%%%%%%%%%%%%%%%%%



\begin{document}

\vspace{\baselineskip}

\vspace{\baselineskip}

\vspace{\baselineskip}

\vspace{\baselineskip}

\vspace{\baselineskip}
\par

\section*{Stay Hydrated}
\addcontentsline{toc}{section}{Stay Hydrated}

\vspace{\baselineskip}

\vspace{\baselineskip}
\paragraph*{Michael BRAVE }
\addcontentsline{toc}{paragraph}{Michael BRAVE }
\paragraph*{42 Staff }
\addcontentsline{toc}{paragraph}{42 Staff }

\vspace{\baselineskip}

\vspace{\baselineskip}

\vspace{\baselineskip}

\vspace{\baselineskip}
\begin{Center}
\textit{Summary: This document contains the subject for Day 00 for the $``$Piscine Swift$"$  from 42}
\end{Center}\par


\vspace{\baselineskip}

\vspace{\baselineskip}

\vspace{\baselineskip}

\vspace{\baselineskip}

\vspace{\baselineskip}

\vspace{\baselineskip}

\vspace{\baselineskip}

\vspace{\baselineskip}


 %%%%%%%%%%%%  Starting New Page here %%%%%%%%%%%%%%

\newpage

\vspace{\baselineskip}
\vspace{\baselineskip}
\section*{Contents}
\addcontentsline{toc}{section}{Contents}

\vspace{\baselineskip}
\subsubsection*{I\hspace*{10pt}\hspace*{10pt}Foreword}
\addcontentsline{toc}{subsubsection}{I\hspace*{10pt}\hspace*{10pt}Foreword}
\subsubsection*{II\hspace*{10pt}\hspace*{10pt}General Instructions}
\addcontentsline{toc}{subsubsection}{II\hspace*{10pt}\hspace*{10pt}General Instructions}
\subsubsection*{III\hspace*{10pt}\hspace*{10pt}Introduction}
\addcontentsline{toc}{subsubsection}{III\hspace*{10pt}\hspace*{10pt}Introduction}
\subsubsection*{IV\hspace*{10pt}\hspace*{10pt}Exercise 00: You Are Awesome ($``$Hello World$"$ )}
\addcontentsline{toc}{subsubsection}{IV\hspace*{10pt}\hspace*{10pt}Exercise 00: You Are Awesome ($``$Hello World$"$ )}
\subsubsection*{V\hspace*{10pt}\hspace*{10pt}Exercise 01: Buttons That Do Things}
\addcontentsline{toc}{subsubsection}{V\hspace*{10pt}\hspace*{10pt}Exercise 01: Buttons That Do Things}
\subsubsection*{VI\hspace*{10pt}\hspace*{10pt}Exercise 02: Join The OO Party}
\addcontentsline{toc}{subsubsection}{VI\hspace*{10pt}\hspace*{10pt}Exercise 02: Join The OO Party}
\subsubsection*{VII\hspace*{10pt}\hspace*{10pt}Exercise 03: Dates $\&$  Times}
\addcontentsline{toc}{subsubsection}{VII\hspace*{10pt}\hspace*{10pt}Exercise 03: Dates $\&$  Times}
\subsubsection*{VIII\hspace*{10pt}\hspace*{10pt}Exercise 04: Stay Hydrated App}
\addcontentsline{toc}{subsubsection}{VIII\hspace*{10pt}\hspace*{10pt}Exercise 04: Stay Hydrated App}
\subsubsection*{XI\hspace*{10pt}\hspace*{10pt}Bonus: Adding Animations $\&$  Graphics}
\addcontentsline{toc}{subsubsection}{XI\hspace*{10pt}\hspace*{10pt}Bonus: Adding Animations $\&$  Graphics}

\vspace{\baselineskip}

\vspace{\baselineskip}

\vspace{\baselineskip}

\vspace{\baselineskip}

\vspace{\baselineskip}

\vspace{\baselineskip}


 %%%%%%%%%%%%  Starting New Page here %%%%%%%%%%%%%%

\newpage

\vspace{\baselineskip}
\vspace{\baselineskip}
\section*{Chapter I}
\addcontentsline{toc}{section}{Chapter I}
\section*{Foreword}
\addcontentsline{toc}{section}{Foreword}

\vspace{\baselineskip}
According to Wikipedia on Drinking\par


\vspace{\baselineskip}
A daily intake of water is required for the normal physiological functioning of the human body. The USDA recommends a daily intake of total water: not necessarily by drinking but by consumption of water contained in other beverages and foods. The recommended intake is 3.7 liters (appx. 1 gallon) per day for an adult male, and 2.7 liters (appx. 0.75 gallon) for an adult female. Other sources, however, claim that a high intake of fresh drinking water, separate and distinct from other sources of moisture, is necessary for good health – eight servings per day of eight fluid ounces (1.8 liters, or 0.5 gallon) is the amount recommended by many nutritionists, although there is no scientific evidence supporting this recommendation.\par


\vspace{\baselineskip}

\vspace{\baselineskip}

\vspace{\baselineskip}

\vspace{\baselineskip}


 %%%%%%%%%%%%  Starting New Page here %%%%%%%%%%%%%%

\newpage

\vspace{\baselineskip}
\vspace{\baselineskip}
\setlength{\parskip}{6.0pt}
\section*{Chapter II}
\addcontentsline{toc}{section}{Chapter II}
\section*{General Instructions}
\addcontentsline{toc}{section}{General Instructions}
\begin{itemize}
	\item Only this document will serve as reference. Do not trust rumors.\par

	\item Read carefully the whole subject before beginning.\par

	\item Watch out! This document could potentially change up to an hour before submission.\par

	\item This project will be corrected by humans only.\par

	\item This course is designed to build on previous days’ concepts, try your hardest to finish everyday.\par

	\item Each day culminates in a portfolio piece, if you finish the day this is something you can use to get hired.\par

	\item When submitting, submit the folder of the Xcode project.\par

	\item Only the work submitted on the repository will be accounted for during peer-2-peer correction.\par

	\item Here is the \href{https://docs.swift.org/swift-book/}{\textcolor[HTML]{1155CC}{\uline{official manual of Swift}}} and the \href{https://developer.apple.com/documentation/swift/swift_standard_library}{\textcolor[HTML]{1155CC}{\uline{Swift Standard Library}}}\par

	\item It is forbidden to use other libraries, packages, pods, etc. Unless otherwise stated in the project.\par

	\item Got a question? Ask your peer on the right. Otherwise, try your peer on the left.\par

	\item You can discuss on the Piscine forum of your Intra!\par

	\item By Odin, by Thor! Use your brain!!!
\end{itemize}\par


\vspace{\baselineskip}

\vspace{\baselineskip}


 %%%%%%%%%%%%  Starting New Page here %%%%%%%%%%%%%%

\newpage

\vspace{\baselineskip}
\vspace{\baselineskip}
\section*{Chapter III}
\addcontentsline{toc}{section}{Chapter III}
\section*{Introduction}
\addcontentsline{toc}{section}{Introduction}

\vspace{\baselineskip}
First we will be learning how to use Xcode, building things in swift without Xcode is possible since it’s open source and works in linux, but for our purposes of building IOS, OSX and other Apple related apps it’s not recommended as a viable solution at this time.\par


\vspace{\baselineskip}
We will build some simple applications today learning how to use Xcode’s Auto-layout, push notifications and API calls. \par


\vspace{\baselineskip}
Each assignment was designed to build on the previous assignments knowledge and culminate in a portfolio piece. If we finish the day we will have a portfolio piece that reminds us to drink enough water everyday and will keep track of how much we’ve consumed. \par


\vspace{\baselineskip}

\vspace{\baselineskip}


 %%%%%%%%%%%%  Starting New Page here %%%%%%%%%%%%%%

\newpage

\vspace{\baselineskip}
\vspace{\baselineskip}
\section*{Chapter IV}
\addcontentsline{toc}{section}{Chapter IV}
\section*{Exercise 00 : You Are Awesome ($``$Hello World$"$ )}
\addcontentsline{toc}{section}{Exercise 00 : You Are Awesome ($``$Hello World$"$ )}

\vspace{\baselineskip}

\vspace{\baselineskip}

\vspace{\baselineskip}


%%%%%%%%%%%%%%%%%%%% Table No: 1 starts here %%%%%%%%%%%%%%%%%%%%


\begin{table}[H]
 			\centering
\begin{tabular}{p{7.3in}}
\hline
%row no:1
\multicolumn{1}{|p{7.3in}|}{\Centering Exercise : 00} \\
\hhline{-}
%row no:2
\multicolumn{1}{|p{7.3in}|}{\Centering You Are Awesome} \\
\hhline{-}
%row no:3
\multicolumn{1}{|p{7.3in}|}{Files to turn in: .xcodeproj and all necessary files} \\
\hhline{-}
%row no:4
\multicolumn{1}{|p{7.3in}|}{Allowed functions : Swift Standard Library, UIKit} \\
\hhline{-}
%row no:5
\multicolumn{1}{|p{7.3in}|}{Notes : n/a} \\
\hhline{-}

\end{tabular}
 \end{table}


%%%%%%%%%%%%%%%%%%%% Table No: 1 ends here %%%%%%%%%%%%%%%%%%%%


\vspace{\baselineskip}

\vspace{\baselineskip}
For this first exercise we are learning to create a project, how to display text to the screen and how to make a button functional. The trick of this assignment is learning how to navigate Xcode.\par


\vspace{\baselineskip}
Create a UIButton, in the main view, that when clicked will display a message in Xcode’s debug console and in a UILabel. \par


\vspace{\baselineskip}

\vspace{\baselineskip}

\vspace{\baselineskip}


 %%%%%%%%%%%%  Starting New Page here %%%%%%%%%%%%%%

\newpage

\vspace{\baselineskip}
\vspace{\baselineskip}
\section*{Chapter V}
\addcontentsline{toc}{section}{Chapter V}
\section*{Exercise 01 : Buttons That Do Things}
\addcontentsline{toc}{section}{Exercise 01 : Buttons That Do Things}

\vspace{\baselineskip}

\vspace{\baselineskip}

\vspace{\baselineskip}


%%%%%%%%%%%%%%%%%%%% Table No: 2 starts here %%%%%%%%%%%%%%%%%%%%


\begin{table}[H]
 			\centering
\begin{tabular}{p{7.3in}}
\hline
%row no:1
\multicolumn{1}{|p{7.3in}|}{\Centering Exercise : 01} \\
\hhline{-}
%row no:2
\multicolumn{1}{|p{7.3in}|}{\Centering Buttons That Do Things} \\
\hhline{-}
%row no:3
\multicolumn{1}{|p{7.3in}|}{Files to turn in: .xcodeproj and all necessary files} \\
\hhline{-}
%row no:4
\multicolumn{1}{|p{7.3in}|}{Allowed functions : Swift Standard Library, UIKit} \\
\hhline{-}
%row no:5
\multicolumn{1}{|p{7.3in}|}{Notes : n/a} \\
\hhline{-}

\end{tabular}
 \end{table}


%%%%%%%%%%%%%%%%%%%% Table No: 2 ends here %%%%%%%%%%%%%%%%%%%%


\vspace{\baselineskip}

\vspace{\baselineskip}
Now it’s not enough to have just one button, we need two. Also Apps should work in both landscape and portrait mode. Lucky for us swift/Xcode has tools like autolayout to help us. \par


\vspace{\baselineskip}
Our app should have as follows:\par

\begin{enumerate}
	\item Two buttons side by side that adapt to the rotation of the screen and change size relationally to the other elements of the page.\par

	\item A UILabel with text describing how awesome you are inside of it, that changes between two fonts depending on which button was pressed.\par

	\item A popup notification that alerts us to when a font was changed.
\end{enumerate}\par


\vspace{\baselineskip}
Pro Tip: Look up StackView, and see if you want to use it with autolayout.\par


\vspace{\baselineskip}


 %%%%%%%%%%%%  Starting New Page here %%%%%%%%%%%%%%

\newpage

\vspace{\baselineskip}
\vspace{\baselineskip}
\section*{Chapter VI}
\addcontentsline{toc}{section}{Chapter VI}
\section*{Exercise 02: Join The OO Party}
\addcontentsline{toc}{section}{Exercise 02: Join The OO Party}

\vspace{\baselineskip}

\vspace{\baselineskip}

\vspace{\baselineskip}


%%%%%%%%%%%%%%%%%%%% Table No: 3 starts here %%%%%%%%%%%%%%%%%%%%


\begin{table}[H]
 			\centering
\begin{tabular}{p{7.3in}}
\hline
%row no:1
\multicolumn{1}{|p{7.3in}|}{\Centering Exercise : 02} \\
\hhline{-}
%row no:2
\multicolumn{1}{|p{7.3in}|}{\Centering Join The OO Party} \\
\hhline{-}
%row no:3
\multicolumn{1}{|p{7.3in}|}{Files to turn in: .xcodeproj and all necessary files} \\
\hhline{-}
%row no:4
\multicolumn{1}{|p{7.3in}|}{Allowed functions : Swift Standard Library, UIKit (CAEmitterLayer / CAEmitterCell)} \\
\hhline{-}
%row no:5
\multicolumn{1}{|p{7.3in}|}{Notes : n/a} \\
\hhline{-}

\end{tabular}
 \end{table}


%%%%%%%%%%%%%%%%%%%% Table No: 3 ends here %%%%%%%%%%%%%%%%%%%%


\vspace{\baselineskip}

\vspace{\baselineskip}
This assignment is to teach us the basics of Object Oriented Programming, or OOP/OO for short. \par


\vspace{\baselineskip}
We are delving into some of the more interesting features of UIKit, dealing with layers, shaders and particles. Not only is this neat to know, but it’s also a great chance to cut our teeth on how objects and inheritance works (which is foundational of what Object Oriented Programming is). \par


\vspace{\baselineskip}
Use the project built previously, but now instead of pushing an update notification we are going to spray confetti all over the screen. You can make the confetti look however you want (snow, puppies, confetti, etc) but there must be 3 separate images, of varying sizes, and varying velocity. One button will create confetti, one button will stop it. \par


\vspace{\baselineskip}
Congratulations, you now know how to use basic animations, use layers, particle systems and the core ideas of object oriented programming. This is worth celebrating. \par


\vspace{\baselineskip}


 %%%%%%%%%%%%  Starting New Page here %%%%%%%%%%%%%%

\newpage

\vspace{\baselineskip}
\vspace{\baselineskip}
\section*{Chapter VII}
\addcontentsline{toc}{section}{Chapter VII}
\section*{Exercise 03: Dates $\&$  Times}
\addcontentsline{toc}{section}{Exercise 03: Dates $\&$  Times}

\vspace{\baselineskip}

\vspace{\baselineskip}

\vspace{\baselineskip}


%%%%%%%%%%%%%%%%%%%% Table No: 4 starts here %%%%%%%%%%%%%%%%%%%%


\begin{table}[H]
 			\centering
\begin{tabular}{p{7.3in}}
\hline
%row no:1
\multicolumn{1}{|p{7.3in}|}{\Centering Exercise : 03} \\
\hhline{-}
%row no:2
\multicolumn{1}{|p{7.3in}|}{\Centering Dates $\&$  Times} \\
\hhline{-}
%row no:3
\multicolumn{1}{|p{7.3in}|}{Files to turn in: .xcodeproj and all necessary files} \\
\hhline{-}
%row no:4
\multicolumn{1}{|p{7.3in}|}{Allowed functions : Swift Standard Library, UIKit} \\
\hhline{-}
%row no:5
\multicolumn{1}{|p{7.3in}|}{Notes : n/a} \\
\hhline{-}

\end{tabular}
 \end{table}


%%%%%%%%%%%%%%%%%%%% Table No: 4 ends here %%%%%%%%%%%%%%%%%%%%


\vspace{\baselineskip}

\vspace{\baselineskip}
We are going to script an event to occur at a specific time, like a minimal alarm clock. A push notification will suffice. \par


\vspace{\baselineskip}

\vspace{\baselineskip}

\vspace{\baselineskip}

\vspace{\baselineskip}


 %%%%%%%%%%%%  Starting New Page here %%%%%%%%%%%%%%

\newpage

\vspace{\baselineskip}
\vspace{\baselineskip}

\vspace{\baselineskip}
\section*{Chapter VIII}
\addcontentsline{toc}{section}{Chapter VIII}
\section*{Exercise 04: Stay Hydrated App}
\addcontentsline{toc}{section}{Exercise 04: Stay Hydrated App}

\vspace{\baselineskip}

\vspace{\baselineskip}

\vspace{\baselineskip}


%%%%%%%%%%%%%%%%%%%% Table No: 5 starts here %%%%%%%%%%%%%%%%%%%%


\begin{table}[H]
 			\centering
\begin{tabular}{p{7.3in}}
\hline
%row no:1
\multicolumn{1}{|p{7.3in}|}{\Centering Exercise : 04} \\
\hhline{-}
%row no:2
\multicolumn{1}{|p{7.3in}|}{\Centering Stay Hydrated App} \\
\hhline{-}
%row no:3
\multicolumn{1}{|p{7.3in}|}{Files to turn in: .xcodeproj and all necessary files} \\
\hhline{-}
%row no:4
\multicolumn{1}{|p{7.3in}|}{Allowed functions : Swift Standard Library, UIKit (CAEmitterLayer / CAEmitterCell)} \\
\hhline{-}
%row no:5
\multicolumn{1}{|p{7.3in}|}{Notes : n/a} \\
\hhline{-}

\end{tabular}
 \end{table}


%%%%%%%%%%%%%%%%%%%% Table No: 5 ends here %%%%%%%%%%%%%%%%%%%%


\vspace{\baselineskip}

\vspace{\baselineskip}
Staying hydrated is important.\par

This assignment is to build an app that uses everything we’ve learned up until this point.\par


\vspace{\baselineskip}
Our App Should:\par

\begin{itemize}
	\item Use push notifications to remind you to drink water (on a timer).\par

	\item Keep track of how much water was drunk that day.\par

	\item Visually represent how much water that is in reference to the daily goal.\par

	\item Reset the amount at the end of the day
\end{itemize}\par


\vspace{\baselineskip}

\vspace{\baselineskip}

\vspace{\baselineskip}


 %%%%%%%%%%%%  Starting New Page here %%%%%%%%%%%%%%

\newpage

\vspace{\baselineskip}
\vspace{\baselineskip}
\section*{Chapter XI}
\addcontentsline{toc}{section}{Chapter XI}
\section*{Bonus : Adding Animations $\&$  Graphics}
\addcontentsline{toc}{section}{Bonus : Adding Animations $\&$  Graphics}

\vspace{\baselineskip}

\vspace{\baselineskip}

\vspace{\baselineskip}


%%%%%%%%%%%%%%%%%%%% Table No: 6 starts here %%%%%%%%%%%%%%%%%%%%


\begin{table}[H]
 			\centering
\begin{tabular}{p{7.3in}}
\hline
%row no:1
\multicolumn{1}{|p{7.3in}|}{\Centering Bonus} \\
\hhline{-}
%row no:2
\multicolumn{1}{|p{7.3in}|}{\Centering Adding Animations $\&$  Graphics} \\
\hhline{-}
%row no:3
\multicolumn{1}{|p{7.3in}|}{Files to turn in: .xcodeproj and all necessary files} \\
\hhline{-}
%row no:4
\multicolumn{1}{|p{7.3in}|}{Allowed functions : Swift Standard Library, UIKit} \\
\hhline{-}
%row no:5
\multicolumn{1}{|p{7.3in}|}{Notes : n/a} \\
\hhline{-}

\end{tabular}
 \end{table}


%%%%%%%%%%%%%%%%%%%% Table No: 6 ends here %%%%%%%%%%%%%%%%%%%%


\vspace{\baselineskip}
Your stay hydrated app is your first IOS portfolio piece, make it look nice, give it the bells and whistles you wish it had, add some neat animations. Make it shine. \par


\printbibliography
\end{document}