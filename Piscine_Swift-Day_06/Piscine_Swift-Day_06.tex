%%%%%%%%%%%%  Generated using docx2latex.com  %%%%%%%%%%%%%%

%%%%%%%%%%%%  v2.0.0-beta  %%%%%%%%%%%%%%

\documentclass[12pt]{report}
\usepackage{amsmath}
\usepackage{latexsym}
\usepackage{amsfonts}
\usepackage[normalem]{ulem}
\usepackage{array}
\usepackage{amssymb}
\usepackage{graphicx}
\usepackage[backend=biber,
style=numeric,
sorting=none,
isbn=false,
doi=false,
url=false,
]{biblatex}\addbibresource{bibliography.bib}

\usepackage{subfig}
\usepackage{wrapfig}
\usepackage{wasysym}
\usepackage{enumitem}
\usepackage{adjustbox}
\usepackage{ragged2e}
\usepackage[svgnames,table]{xcolor}
\usepackage{tikz}
\usepackage{longtable}
\usepackage{changepage}
\usepackage{setspace}
\usepackage{hhline}
\usepackage{multicol}
\usepackage{tabto}
\usepackage{float}
\usepackage{multirow}
\usepackage{makecell}
\usepackage{fancyhdr}
\usepackage[toc,page]{appendix}
\usepackage[hidelinks]{hyperref}
\usetikzlibrary{shapes.symbols,shapes.geometric,shadows,arrows.meta}
\tikzset{>={Latex[width=1.5mm,length=2mm]}}
\usepackage{flowchart}\usepackage[paperheight=11.0in,paperwidth=8.5in,left=0.5in,right=0.5in,top=0.5in,bottom=0.5in,headheight=1in]{geometry}
\usepackage[utf8]{inputenc}
\usepackage[T1]{fontenc}
\TabPositions{0.5in,1.0in,1.5in,2.0in,2.5in,3.0in,3.5in,4.0in,4.5in,5.0in,5.5in,6.0in,6.5in,7.0in,}

\urlstyle{same}


 %%%%%%%%%%%%  Set Depths for Sections  %%%%%%%%%%%%%%

% 1) Section
% 1.1) SubSection
% 1.1.1) SubSubSection
% 1.1.1.1) Paragraph
% 1.1.1.1.1) Subparagraph


\setcounter{tocdepth}{5}
\setcounter{secnumdepth}{5}


 %%%%%%%%%%%%  Set Depths for Nested Lists created by \begin{enumerate}  %%%%%%%%%%%%%%


\setlistdepth{9}
\renewlist{enumerate}{enumerate}{9}
		\setlist[enumerate,1]{label=\arabic*)}
		\setlist[enumerate,2]{label=\alph*)}
		\setlist[enumerate,3]{label=(\roman*)}
		\setlist[enumerate,4]{label=(\arabic*)}
		\setlist[enumerate,5]{label=(\Alph*)}
		\setlist[enumerate,6]{label=(\Roman*)}
		\setlist[enumerate,7]{label=\arabic*}
		\setlist[enumerate,8]{label=\alph*}
		\setlist[enumerate,9]{label=\roman*}

\renewlist{itemize}{itemize}{9}
		\setlist[itemize]{label=$\cdot$}
		\setlist[itemize,1]{label=\textbullet}
		\setlist[itemize,2]{label=$\circ$}
		\setlist[itemize,3]{label=$\ast$}
		\setlist[itemize,4]{label=$\dagger$}
		\setlist[itemize,5]{label=$\triangleright$}
		\setlist[itemize,6]{label=$\bigstar$}
		\setlist[itemize,7]{label=$\blacklozenge$}
		\setlist[itemize,8]{label=$\prime$}

\setlength{\topsep}{0pt}\setlength{\parindent}{0pt}

 %%%%%%%%%%%%  This sets linespacing (verticle gap between Lines) Default=1 %%%%%%%%%%%%%%


\renewcommand{\arraystretch}{1.3}


%%%%%%%%%%%%%%%%%%%% Document code starts here %%%%%%%%%%%%%%%%%%%%



\begin{document}

\vspace{\baselineskip}

\vspace{\baselineskip}

\vspace{\baselineskip}

\vspace{\baselineskip}

\vspace{\baselineskip}
\par

\section*{Earth, Water $\&$  Fire BNB}
\addcontentsline{toc}{section}{Earth, Water $\&$  Fire BNB}

\vspace{\baselineskip}

\vspace{\baselineskip}
\paragraph*{Michael BRAVE }
\addcontentsline{toc}{paragraph}{Michael BRAVE }
\paragraph*{42 Staff }
\addcontentsline{toc}{paragraph}{42 Staff }

\vspace{\baselineskip}

\vspace{\baselineskip}

\vspace{\baselineskip}

\vspace{\baselineskip}
\begin{Center}
\textit{Summary: This document contains the subject for Day for the $``$Piscine Swift$"$  from 42}
\end{Center}\par


\vspace{\baselineskip}

\vspace{\baselineskip}

\vspace{\baselineskip}

\vspace{\baselineskip}

\vspace{\baselineskip}

\vspace{\baselineskip}

\vspace{\baselineskip}

\vspace{\baselineskip}


 %%%%%%%%%%%%  Starting New Page here %%%%%%%%%%%%%%

\newpage

\vspace{\baselineskip}
\vspace{\baselineskip}
\section*{Contents}
\addcontentsline{toc}{section}{Contents}

\vspace{\baselineskip}
\subsubsection*{I\hspace*{10pt}\hspace*{10pt}Foreword}
\addcontentsline{toc}{subsubsection}{I\hspace*{10pt}\hspace*{10pt}Foreword}
\subsubsection*{II\hspace*{10pt}\hspace*{10pt}General Instructions}
\addcontentsline{toc}{subsubsection}{II\hspace*{10pt}\hspace*{10pt}General Instructions}
\subsubsection*{III\hspace*{10pt}\hspace*{10pt}Introduction}
\addcontentsline{toc}{subsubsection}{III\hspace*{10pt}\hspace*{10pt}Introduction}
\subsubsection*{IV\hspace*{10pt}\hspace*{10pt}Exercise 00: Map API Calls}
\addcontentsline{toc}{subsubsection}{IV\hspace*{10pt}\hspace*{10pt}Exercise 00: Map API Calls}
\subsubsection*{V\hspace*{10pt}\hspace*{10pt}Exercise 01: Where Am I}
\addcontentsline{toc}{subsubsection}{V\hspace*{10pt}\hspace*{10pt}Exercise 01: Where Am I}
\subsubsection*{VI\hspace*{10pt}\hspace*{10pt}Exercise 02: Where Is Home}
\addcontentsline{toc}{subsubsection}{VI\hspace*{10pt}\hspace*{10pt}Exercise 02: Where Is Home}
\subsubsection*{VII\hspace*{10pt}\hspace*{10pt}Exercise 03: Charting Routes}
\addcontentsline{toc}{subsubsection}{VII\hspace*{10pt}\hspace*{10pt}Exercise 03: Charting Routes}
\subsubsection*{VIII\hspace*{10pt}\hspace*{10pt}Exercise 04: Making A Storefront}
\addcontentsline{toc}{subsubsection}{VIII\hspace*{10pt}\hspace*{10pt}Exercise 04: Making A Storefront}
\subsubsection*{XI\hspace*{10pt}\hspace*{10pt}Exercise 05: Earth BNB}
\addcontentsline{toc}{subsubsection}{XI\hspace*{10pt}\hspace*{10pt}Exercise 05: Earth BNB}
\subsubsection*{X\hspace*{10pt}\hspace*{10pt}Bonus: Make It Look Professional}
\addcontentsline{toc}{subsubsection}{X\hspace*{10pt}\hspace*{10pt}Bonus: Make It Look Professional}

\vspace{\baselineskip}

\vspace{\baselineskip}

\vspace{\baselineskip}

\vspace{\baselineskip}

\vspace{\baselineskip}

\vspace{\baselineskip}


 %%%%%%%%%%%%  Starting New Page here %%%%%%%%%%%%%%

\newpage

\vspace{\baselineskip}
\vspace{\baselineskip}
\section*{Chapter I}
\addcontentsline{toc}{section}{Chapter I}
\section*{Foreword}
\addcontentsline{toc}{section}{Foreword}
From The Map Is Not The Territory from Farnham Street\par


\vspace{\baselineskip}
The map of reality is not reality. Even the best maps are imperfect. That’s because they are reductions of what they represent. If a map were to represent the territory with perfect fidelity, it would no longer be a reduction and thus would no longer be useful to us. A map can also be a snapshot of a point in time, representing something that no longer exists. This is important to keep in mind as we think through problems and make better decisions.\par


\vspace{\baselineskip}
The Relationship Between Map and Territory\par

In 1931, in New Orleans, Louisiana, mathematician Alfred Korzybski presented a paper on mathematical semantics. To the non-technical reader, most of the paper reads like an abstruse argument on the relationship of mathematics to human language, and of both to physical reality. Important stuff certainly, but not necessarily immediately useful for the layperson.\par


\vspace{\baselineskip}
However, in his string of arguments on the structure of language, Korzybski introduced and popularized the idea that the map is not the territory. In other words, the description of the thing is not the thing itself. The model is not reality. The abstraction is not the abstracted. This has enormous practical consequences.\par


\vspace{\baselineskip}
In Korzybski’s words:\par


\vspace{\baselineskip}
\begin{adjustwidth}{0.5in}{0.0in}
A. A map may have a structure similar or dissimilar to the structure of the territory.\par

\end{adjustwidth}


\vspace{\baselineskip}
\begin{adjustwidth}{0.5in}{0.0in}
B. Two similar structures have similar ‘logical’ characteristics. Thus, if in a correct map, Dresden is given as between Paris and Warsaw, a similar relation is found in the actual territory.\par

\end{adjustwidth}


\vspace{\baselineskip}
\begin{adjustwidth}{0.5in}{0.0in}
C. A map is not the actual territory.\par

\end{adjustwidth}


\vspace{\baselineskip}
\begin{adjustwidth}{0.5in}{0.0in}
D. An ideal map would contain the map of the map, the map of the map of the map, etc., endlessly$ \ldots $ We may call this characteristic self-reflexiveness.\par

\end{adjustwidth}


\vspace{\baselineskip}
Maps are necessary, but flawed. (By maps, we mean any abstraction of reality, including descriptions, theories, models, etc.) The problem with a map is not simply that it is an abstraction; we need abstraction. A map with the scale of one mile to one mile would not have the problems that maps have, nor would it be helpful in any way.\par


\vspace{\baselineskip}
To solve this problem, the mind creates maps of reality in order to understand it, because the only way we can process the complexity of reality is through abstraction. But frequently, we don’t understand our maps or their limits. In fact, we are so reliant on abstraction that we will frequently use an incorrect model simply because we feel any model is preferable to no model. (Reminding one of the drunk looking for his keys under the streetlight because $``$That’s where the light is!$"$ )\par


\vspace{\baselineskip}
Even the best and most useful maps suffer from limitations, and Korzybski gives us a few to explore: (A.) The map could be incorrect without us realizing it; (B.) The map is, by necessity, a reduction of the actual thing, a process in which you lose certain important information; and (C.) A map needs interpretation, a process that can cause major errors. (The only way to truly solve the last would be an endless chain of maps-of-maps, which he called self-reflexiveness.)\par


\vspace{\baselineskip}
With the aid of modern psychology, we also see another issue: the human brain takes great leaps and shortcuts in order to make sense of its surroundings. As Charlie Munger has pointed out, a good idea and the human mind act something like the sperm and the egg — after the first good idea gets in, the door closes. This makes the map-territory problem a close cousin of man-with-a-hammer tendency.\par


\vspace{\baselineskip}
This tendency is obviously problematic in our effort to simplify reality. When we see a powerful model work well, we tend to over-apply it, using it in non-analogous situations. We have trouble delimiting its usefulness, which causes errors.\par


\vspace{\baselineskip}

\vspace{\baselineskip}

\vspace{\baselineskip}

\vspace{\baselineskip}

\vspace{\baselineskip}

\vspace{\baselineskip}

\vspace{\baselineskip}

\vspace{\baselineskip}


 %%%%%%%%%%%%  Starting New Page here %%%%%%%%%%%%%%

\newpage

\vspace{\baselineskip}
\vspace{\baselineskip}
\section*{Chapter II}
\addcontentsline{toc}{section}{Chapter II}
\section*{General Instructions}
\addcontentsline{toc}{section}{General Instructions}
\begin{itemize}
	\item Only this document will serve as reference. Do not trust rumors.\par

	\item Read carefully the whole subject before beginning.\par

	\item Watch out! This document could potentially change up to an hour before submission.\par

	\item This project will be corrected by humans only.\par

	\item This course is designed to build on previous days’ concepts, try your hardest to finish everyday.\par

	\item Each day culminates in a portfolio piece, if you finish the day this is something you can use to get hired.\par

	\item When submitting, submit the folder of the Xcode project.\par

	\item Only the work submitted on the repository will be accounted for during peer-2-peer correction.\par

	\item Here it is the \href{https://docs.swift.org/swift-book/}{\textcolor[HTML]{1155CC}{\uline{official manual of Swift}}} and the \href{https://developer.apple.com/documentation/swift/swift_standard_library}{\textcolor[HTML]{1155CC}{\uline{Swift Standard Library}}}\par

	\item It is forbidden to use other libraries, packages, pods, etc. Unless otherwise stated in the project.\par

	\item Got a question? Ask your peer on the right. Otherwise, try your peer on the left.\par

	\item You can discuss on the Piscine forum of your Intra!\par

	\item By Odin, by Thor! Use your brain!!!
\end{itemize}\par


\vspace{\baselineskip}

\vspace{\baselineskip}


 %%%%%%%%%%%%  Starting New Page here %%%%%%%%%%%%%%

\newpage

\vspace{\baselineskip}
\vspace{\baselineskip}
\section*{Chapter III}
\addcontentsline{toc}{section}{Chapter III}
\section*{Introduction}
\addcontentsline{toc}{section}{Introduction}
Today is all about maps and to learn about it we will build an AirBNB clone complete with directions, profiles and a rent it button. \par


\vspace{\baselineskip}


 %%%%%%%%%%%%  Starting New Page here %%%%%%%%%%%%%%

\newpage

\vspace{\baselineskip}
\vspace{\baselineskip}
\section*{Chapter IV}
\addcontentsline{toc}{section}{Chapter IV}
\section*{Exercise 00 : Map API Calls}
\addcontentsline{toc}{section}{Exercise 00 : Map API Calls}

\vspace{\baselineskip}

\vspace{\baselineskip}

\vspace{\baselineskip}


%%%%%%%%%%%%%%%%%%%% Table No: 1 starts here %%%%%%%%%%%%%%%%%%%%


\begin{table}[H]
 			\centering
\begin{tabular}{p{7.3in}}
\hline
%row no:1
\multicolumn{1}{|p{7.3in}|}{\Centering Exercise : 00} \\
\hhline{-}
%row no:2
\multicolumn{1}{|p{7.3in}|}{\Centering Map API Calls} \\
\hhline{-}
%row no:3
\multicolumn{1}{|p{7.3in}|}{Files to turn in: .xcodeproj and all necessary files} \\
\hhline{-}
%row no:4
\multicolumn{1}{|p{7.3in}|}{Allowed functions : Swift Standard Library, UIKit, MapKit} \\
\hhline{-}
%row no:5
\multicolumn{1}{|p{7.3in}|}{Notes : n/a} \\
\hhline{-}

\end{tabular}
 \end{table}


%%%%%%%%%%%%%%%%%%%% Table No: 1 ends here %%%%%%%%%%%%%%%%%%%%


\vspace{\baselineskip}
We are learning about maps, this assignment is to get MapKit displayed on screen, displaying geographic content. \par


\vspace{\baselineskip}

\vspace{\baselineskip}


 %%%%%%%%%%%%  Starting New Page here %%%%%%%%%%%%%%

\newpage

\vspace{\baselineskip}
\vspace{\baselineskip}
\section*{Chapter V}
\addcontentsline{toc}{section}{Chapter V}
\section*{Exercise 01 : Where Am I?}
\addcontentsline{toc}{section}{Exercise 01 : Where Am I?}

\vspace{\baselineskip}

\vspace{\baselineskip}

\vspace{\baselineskip}


%%%%%%%%%%%%%%%%%%%% Table No: 2 starts here %%%%%%%%%%%%%%%%%%%%


\begin{table}[H]
 			\centering
\begin{tabular}{p{7.3in}}
\hline
%row no:1
\multicolumn{1}{|p{7.3in}|}{\Centering Exercise : 01} \\
\hhline{-}
%row no:2
\multicolumn{1}{|p{7.3in}|}{\Centering Where Am I?} \\
\hhline{-}
%row no:3
\multicolumn{1}{|p{7.3in}|}{Files to turn in: .xcodeproj and all necessary files} \\
\hhline{-}
%row no:4
\multicolumn{1}{|p{7.3in}|}{Allowed functions : Swift Standard Library, UIKit, MapKit} \\
\hhline{-}
%row no:5
\multicolumn{1}{|p{7.3in}|}{Notes : n/a} \\
\hhline{-}

\end{tabular}
 \end{table}


%%%%%%%%%%%%%%%%%%%% Table No: 2 ends here %%%%%%%%%%%%%%%%%%%%


\vspace{\baselineskip}
Now we have to make markers on our map, to begin we should show our current location. \par


\vspace{\baselineskip}


 %%%%%%%%%%%%  Starting New Page here %%%%%%%%%%%%%%

\newpage

\vspace{\baselineskip}
\vspace{\baselineskip}
\section*{Chapter VI}
\addcontentsline{toc}{section}{Chapter VI}
\section*{Exercise 02: Where Is Home}
\addcontentsline{toc}{section}{Exercise 02: Where Is Home}

\vspace{\baselineskip}

\vspace{\baselineskip}

\vspace{\baselineskip}


%%%%%%%%%%%%%%%%%%%% Table No: 3 starts here %%%%%%%%%%%%%%%%%%%%


\begin{table}[H]
 			\centering
\begin{tabular}{p{7.3in}}
\hline
%row no:1
\multicolumn{1}{|p{7.3in}|}{\Centering Exercise : 02} \\
\hhline{-}
%row no:2
\multicolumn{1}{|p{7.3in}|}{\Centering Where Is Home} \\
\hhline{-}
%row no:3
\multicolumn{1}{|p{7.3in}|}{Files to turn in: .xcodeproj and all necessary files} \\
\hhline{-}
%row no:4
\multicolumn{1}{|p{7.3in}|}{Allowed functions : Swift Standard Library, UIKit, MapKit} \\
\hhline{-}
%row no:5
\multicolumn{1}{|p{7.3in}|}{Notes : n/a} \\
\hhline{-}

\end{tabular}
 \end{table}


%%%%%%%%%%%%%%%%%%%% Table No: 3 ends here %%%%%%%%%%%%%%%%%%%%


\vspace{\baselineskip}
For this assignment we want to input our own address and have an accurate marker that represents our location based on the address provided. \par



 %%%%%%%%%%%%  Starting New Page here %%%%%%%%%%%%%%

\newpage

\vspace{\baselineskip}
\vspace{\baselineskip}
\section*{Chapter VII}
\addcontentsline{toc}{section}{Chapter VII}
\section*{Exercise 03: Charting Routes}
\addcontentsline{toc}{section}{Exercise 03: Charting Routes}

\vspace{\baselineskip}

\vspace{\baselineskip}

\vspace{\baselineskip}


%%%%%%%%%%%%%%%%%%%% Table No: 4 starts here %%%%%%%%%%%%%%%%%%%%


\begin{table}[H]
 			\centering
\begin{tabular}{p{7.3in}}
\hline
%row no:1
\multicolumn{1}{|p{7.3in}|}{\Centering Exercise : 03} \\
\hhline{-}
%row no:2
\multicolumn{1}{|p{7.3in}|}{\Centering Charting Routes} \\
\hhline{-}
%row no:3
\multicolumn{1}{|p{7.3in}|}{Files to turn in: .xcodeproj and all necessary files} \\
\hhline{-}
%row no:4
\multicolumn{1}{|p{7.3in}|}{Allowed functions : Swift Standard Library, UIKit, MapKit} \\
\hhline{-}
%row no:5
\multicolumn{1}{|p{7.3in}|}{Notes : n/a} \\
\hhline{-}

\end{tabular}
 \end{table}


%%%%%%%%%%%%%%%%%%%% Table No: 4 ends here %%%%%%%%%%%%%%%%%%%%


\vspace{\baselineskip}
Now we want to chart courses between two points of interest. Try using our house and the nearest grocery store. Now try three other addresses. \par


\vspace{\baselineskip}

\vspace{\baselineskip}

\vspace{\baselineskip}

\vspace{\baselineskip}


 %%%%%%%%%%%%  Starting New Page here %%%%%%%%%%%%%%

\newpage

\vspace{\baselineskip}
\vspace{\baselineskip}

\vspace{\baselineskip}
\section*{Chapter VIII}
\addcontentsline{toc}{section}{Chapter VIII}
\section*{Exercise 04: Making A Storefront}
\addcontentsline{toc}{section}{Exercise 04: Making A Storefront}

\vspace{\baselineskip}

\vspace{\baselineskip}

\vspace{\baselineskip}


%%%%%%%%%%%%%%%%%%%% Table No: 5 starts here %%%%%%%%%%%%%%%%%%%%


\begin{table}[H]
 			\centering
\begin{tabular}{p{7.3in}}
\hline
%row no:1
\multicolumn{1}{|p{7.3in}|}{\Centering Exercise : 04} \\
\hhline{-}
%row no:2
\multicolumn{1}{|p{7.3in}|}{\Centering Making A Storefront} \\
\hhline{-}
%row no:3
\multicolumn{1}{|p{7.3in}|}{Files to turn in: .xcodeproj and all necessary files} \\
\hhline{-}
%row no:4
\multicolumn{1}{|p{7.3in}|}{Allowed functions : Swift Standard Library, UIKit, MapKit} \\
\hhline{-}
%row no:5
\multicolumn{1}{|p{7.3in}|}{Notes : n/a} \\
\hhline{-}

\end{tabular}
 \end{table}


%%%%%%%%%%%%%%%%%%%% Table No: 5 ends here %%%%%%%%%%%%%%%%%%%%


\vspace{\baselineskip}
Now we want to build a storefront. We want to make pages that display information about potential rental locations, this includes, pricing, photos, descriptions and contact info. Make about four fake locations nearby. \par


\vspace{\baselineskip}


 %%%%%%%%%%%%  Starting New Page here %%%%%%%%%%%%%%

\newpage

\vspace{\baselineskip}
\vspace{\baselineskip}

\vspace{\baselineskip}
\section*{Chapter XI}
\addcontentsline{toc}{section}{Chapter XI}
\section*{Exercise 05: Earth BNB}
\addcontentsline{toc}{section}{Exercise 05: Earth BNB}

\vspace{\baselineskip}

\vspace{\baselineskip}

\vspace{\baselineskip}


%%%%%%%%%%%%%%%%%%%% Table No: 6 starts here %%%%%%%%%%%%%%%%%%%%


\begin{table}[H]
 			\centering
\begin{tabular}{p{7.3in}}
\hline
%row no:1
\multicolumn{1}{|p{7.3in}|}{\Centering Exercise : 04} \\
\hhline{-}
%row no:2
\multicolumn{1}{|p{7.3in}|}{\Centering Earth BNB} \\
\hhline{-}
%row no:3
\multicolumn{1}{|p{7.3in}|}{Files to turn in: .xcodeproj and all necessary files} \\
\hhline{-}
%row no:4
\multicolumn{1}{|p{7.3in}|}{Allowed functions : Swift Standard Library, UIKit, MapKit} \\
\hhline{-}
%row no:5
\multicolumn{1}{|p{7.3in}|}{Notes : n/a} \\
\hhline{-}

\end{tabular}
 \end{table}


%%%%%%%%%%%%%%%%%%%% Table No: 6 ends here %%%%%%%%%%%%%%%%%%%%


\vspace{\baselineskip}
Putting it all together. Now we are going to make an AirBNB clone, so let’s call it EarthBNB. We will not implement the backend to actually rent a location or accept payments but all other functionality should be present. Ratings and reviews, locations displayed on map, amenities listed, photos and descriptions. We should have basic info previews on click on the map with the option to go the the store listing pages for more details should we choose to. We should be able to go back to the map and forward to the last viewed location page as needed. \par


\vspace{\baselineskip}

\vspace{\baselineskip}

\vspace{\baselineskip}


 %%%%%%%%%%%%  Starting New Page here %%%%%%%%%%%%%%

\newpage

\vspace{\baselineskip}
\vspace{\baselineskip}

\vspace{\baselineskip}

\vspace{\baselineskip}
\section*{Chapter X}
\addcontentsline{toc}{section}{Chapter X}
\section*{Bonus : Make It Look Professional}
\addcontentsline{toc}{section}{Bonus : Make It Look Professional}

\vspace{\baselineskip}

\vspace{\baselineskip}

\vspace{\baselineskip}


%%%%%%%%%%%%%%%%%%%% Table No: 7 starts here %%%%%%%%%%%%%%%%%%%%


\begin{table}[H]
 			\centering
\begin{tabular}{p{7.3in}}
\hline
%row no:1
\multicolumn{1}{|p{7.3in}|}{\Centering Bonus} \\
\hhline{-}
%row no:2
\multicolumn{1}{|p{7.3in}|}{\Centering Make It Look Professional} \\
\hhline{-}
%row no:3
\multicolumn{1}{|p{7.3in}|}{Files to turn in: .xcodeproj and all necessary files} \\
\hhline{-}
%row no:4
\multicolumn{1}{|p{7.3in}|}{Allowed functions : Swift Standard Library, UIKit} \\
\hhline{-}
%row no:5
\multicolumn{1}{|p{7.3in}|}{Notes : n/a} \\
\hhline{-}

\end{tabular}
 \end{table}


%%%%%%%%%%%%%%%%%%%% Table No: 7 ends here %%%%%%%%%%%%%%%%%%%%


\vspace{\baselineskip}

\vspace{\baselineskip}
Add some polish, create some design. Go nuts, but just make it look better than what we already have, this will increase the value of your portfolio. \par


\vspace{\baselineskip}

\printbibliography
\end{document}